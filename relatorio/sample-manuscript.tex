%%
%% This is file `sample-manuscript.tex',
%% generated with the docstrip utility.
%%
%% The original source files were:
%%
%% samples.dtx  (with options: `all,proceedings,bibtex,manuscript')
%% 
%% IMPORTANT NOTICE:
%% 
%% For the copyright see the source file.
%% 
%% Any modified versions of this file must be renamed
%% with new filenames distinct from sample-manuscript.tex.
%% 
%% For distribution of the original source see the terms
%% for copying and modification in the file samples.dtx.
%% 
%% This generated file may be distributed as long as the
%% original source files, as listed above, are part of the
%% same distribution. (The sources need not necessarily be
%% in the same archive or directory.)
%%
%%
%% Commands for TeXCount
%TC:macro \cite [option:text,text]
%TC:macro \citep [option:text,text]
%TC:macro \citet [option:text,text]
%TC:envir table 0 1
%TC:envir table* 0 1
%TC:envir tabular [ignore] word
%TC:envir displaymath 0 word
%TC:envir math 0 word
%TC:envir comment 0 0
%%
%% The first command in your LaTeX source must be the \documentclass
%% command.
%%
%% For submission and review of your manuscript please change the
%% command to \documentclass[manuscript, screen, review]{acmart}.
%%
%% When submitting camera ready or to TAPS, please change the command
%% to \documentclass[sigconf]{acmart} or whichever template is required
%% for your publication.
%%
%%
\documentclass[sigplan]{acmart}
\settopmatter{printacmref=false} % Removes citation information below abstract
\renewcommand\footnotetextcopyrightpermission[1]{} % removes footnote with conference information in first column
\usepackage{subcaption}
\usepackage{caption}
\usepackage{array}
\usepackage{multirow}
\usepackage{float}

\begin{document}

%%
%% The "title" command has an optional parameter,
%% allowing the author to define a "short title" to be used in page headers.
\title{Complex Data Analisys - MathOverflow}

\author{João Oliveira}
\affiliation{%
  \institution{Faculdade de Ciências da Universidade de Lisboa}
  \city{Lisboa}
  \country{Portugal}
}
\email{fc56908@alunos.fc.ul.pt}

\author{Pedro Gomes}
\affiliation{%
  \institution{Faculdade de Ciências da Universidade de Lisboa}
  \city{Lisboa}
  \country{Portugal}
}
\email{fc58167@alunos.fc.ul.pt}

\author{Tiago Frade}
\affiliation{%
  \institution{Faculdade de Ciências da Universidade de Lisboa}
  \city{Lisboa}
  \country{Portugal}
}
\email{fc59795@alunos.fc.ul.pt}

\keywords{Network Analysis, MathOverflow, SNAP Dataset, Social Networks, Graph Mining, Community Detection}

\pagestyle{plain} % removes running headers

%%
%% The abstract is a short summary of the work to be presented in the
%% article.
\begin{abstract}
  A clear and well-documented \LaTeX\ document is presented as an
  article formatted for publication by ACM in a conference proceedings
  or journal publication. Based on the ``acmart'' document class, this
  article presents and explains many of the common variations, as well
  as many of the formatting elements an author may use in the
  preparation of the documentation of their work.
\end{abstract}

\maketitle

%%
%% The code below is generated by the tool at http://dl.acm.org/ccs.cfm.
%% Please copy and paste the code instead of the example below.
%%
\begin{CCSXML}
<ccs2012>
 <concept>
  <concept_id>00000000.0000000.0000000</concept_id>
  <concept_desc>Do Not Use This Code, Generate the Correct Terms for Your Paper</concept_desc>
  <concept_significance>500</concept_significance>
 </concept>
 <concept>
  <concept_id>00000000.00000000.00000000</concept_id>
  <concept_desc>Do Not Use This Code, Generate the Correct Terms for Your Paper</concept_desc>
  <concept_significance>300</concept_significance>
 </concept>
 <concept>
  <concept_id>00000000.00000000.00000000</concept_id>
  <concept_desc>Do Not Use This Code, Generate the Correct Terms for Your Paper</concept_desc>
  <concept_significance>100</concept_significance>
 </concept>
 <concept>
  <concept_id>00000000.00000000.00000000</concept_id>
  <concept_desc>Do Not Use This Code, Generate the Correct Terms for Your Paper</concept_desc>
  <concept_significance>100</concept_significance>
 </concept>
</ccs2012>
\end{CCSXML}

\ccsdesc[500]{Do Not Use This Code~Generate the Correct Terms for Your Paper}
\ccsdesc[300]{Do Not Use This Code~Generate the Correct Terms for Your Paper}
\ccsdesc{Do Not Use This Code~Generate the Correct Terms for Your Paper}
\ccsdesc[100]{Do Not Use This Code~Generate the Correct Terms for Your Paper}


\section{Introduction}\label{sec:introduction}
    \begin{itemize}
        \item Explain the motivation of the project
        \item The importance of analysing real-world social networks
        \item Why MathOverflow is a suitable case study
        \item Introduce the goals of the assignment: subset creation, metric computation, comparison with literature, and community visualization
    \end{itemize}
    
    \paragraph{Project Repository.}
        All source code, data subsets, notebooks, and Gephi visualizations used in this report are publicly available at:\\
        \url{https://github.com/jotanmiguel/ProjetoADC-Grupo12}.


\section{Dataset Description}\label{sec:dataset-description}
    Describe the SX-MathOverflow dataset from SNAP:
    \begin{itemize}
        \item Source and structure of the dataset.
        \item Meaning of the \texttt{sx-mathoverflow} and its temporal versions.
        \item Focus on the \texttt{sx-mathoverflow-a2q} graph (answer-to-question).
        \item Basic statistics of the raw dataset before filtering.
    \end{itemize}

    From the Stanford Network Analysis Project (SNAP)\footnote{https://snap.stanford.edu/data/sx-mathoverflow.html}, we've chosen \texttt{sx-mathoverflow-a2q} as our dataset. MathOverflow\footnote{http://MathOverflow.net} is a question-and-answer (Q\&A) platform targeted at professional mathematicians and researchers, where users post research-level problems and exchange knowledge. The SNAP dataset captures interactions between users extracted from the public Stack Exchange data dumps \cite{socialMathoveflow2013}.

    \subsection{Structure of the SX-MathOverflow Dataset}

        The full dataset represents MathOverflow as a directed temporal network. Each node corresponds to a user account, and each temporal edge is a triplet $(u, v, t)$ indicating that user $u$ interacted with user $v$ at timestamp $t$. Interactions include:
        \begin{itemize}
            \item answering another user's question,
            \item commenting on a question,
            \item commenting on an answer.
        \end{itemize}
        
        The complete temporal dataset contains \textbf{24,818 nodes} and \textbf{506,550 time-stamped interactions} in a period of 2,350 days. A static projection of all interactions produces a directed graph with 239,978 edges.
        
        SNAP additionally provides three derived interaction layers, each isolating a specific type of user activity:
        \begin{itemize}
            \item \texttt{sx-mathoverflow-a2q}: answer-to-question interactions,
            \item \texttt{sx-mathoverflow-c2q}: comment-to-question interactions,
            \item \texttt{sx-mathoverflow-c2a}: comment-to-answer interactions.
        \end{itemize}

    \subsection{The \texttt{sx-mathoverflow-a2q} Answer-to-Question Network}

        This work focuses on the \texttt{sx-mathoverflow-a2q} temporal network, which is a curated subset of the full dataset containing only direct answer events. An edge $(u, v, t)$ in this graph indicates that user $u$ provided an answer to a question originally posted by user $v$ at time $t$, representing a direct knowledge-transfer event.
        
        The \texttt{a2q} layer contains:
        \begin{itemize}
            \item \textbf{21,688 nodes},
            \item \textbf{107,581 temporal answer events},
            \item a static directed graph with \textbf{90,489 edges}.
        \end{itemize}
        
        Compared to the full MathOverflow network, this subset removes comments and meta-discussion, retaining only the core problem-solving interactions. Prior literature treats answer exchanges as the most meaningful signal of expert collaboration, making this layer particularly well-suited for structural analysis.
        
        \subsection{Raw Dataset Statistics Before Filtering}
        
        Before applying any filtering (temporal or activity-based), the raw \texttt{a2q} dataset exhibits the following properties:
        \begin{itemize}
            \item node count: 21,688,
            \item temporal edge count: 107,581,
            \item static edge count: 90,489,
            \item directed interactions,
            \item timestamps spanning more than five years.
        \end{itemize}
        
        These initial statistics serve as the baseline against which the effects of our subsequent filtering steps (Section~\ref{sec:subset}) are evaluated.

\section{Subset Creation and Justification}\label{sec:subset-creation}

The original Math Overflow interaction dataset ($N_{\text{nodes}} = 24,818$; $N_{\text{temporal edges}} = 506,550$) is extensive. To focus the analysis on the most representative dynamics of the community and ensure computational tractability for complex network metrics, a two-step procedure was implemented to create a representative subset.

The adopted methodology combines: (1) Temporal Filtering, (2) User Activity Filtering.

\subsection{Temporal Filtering}

The first step involved isolating a specific time window that captures a phase of high activity and network consolidation within the platform. The analysis of the temporal evolution of the number of interactions (temporal edges) over time is presented in Figure \ref{fig:minha_figura}.

\begin{figure}[h]
  \centering
  \includegraphics[width=\linewidth]{temporal_dataset.png}
  \caption{Temporal evolution of the volume of interactions (edges) in the Math Overflow dataset. The graph clearly shows a spike in growth between 2010 and 2012.}
  \label{fig:minha_figura}
\end{figure}

The graph clearly shows a significant spike in interaction volume between 2010 and 2012, followed by a period of relative stabilization. Based on this observation, the initial subset was delimited to the period from January 1, 2010, to December 31, 2012.

This initial action reduced the dataset to 12,470 nodes (50\% reduction) and 266,094 temporal edges (48\% reduction).

\subsection{User Activity Filtering}\label{sec:user-activity-filter}

The second step applied an activity filter to isolate the core participants of the community, thereby excluding inactive or "one-shot" users whose peripheral activity could distort structural metrics.

\begin{itemize}
    \item \textbf{Procedure:} The total degree (sum of in-degree and out-degree) was calculated for all users within the temporal subset.
    \item \textbf{Criterion:} Only users with a total degree greater than 25 interactions were retained.
    \item \textbf{Justification:} This minimum threshold defines a core set of active contributors. It ensures that the resulting induced subgraph reflects repeated and meaningful interactions, where knowledge transfer and expertise are most likely to be exhibited.
\end{itemize}

The final edge list retained only interactions where both the source ($src$) and the target ($dst$) nodes satisfied this minimum activity criterion.

\subsubsection{\textbf{Filtered Graph Statistics}}

After applying the activity filter, the graph was constructed, containing:
\begin{itemize}
    \item \textbf{Nodes (Users):} \textbf{2457}
    \item \textbf{Edges (Static Interactions):} \textbf{88294} 
\end{itemize}

With the final subset, we were able to analyze it´s properties and it was revealed that this subset is completely connected, it´s largest weakly connected component represents all of the subset (2457 nodes).

We can see a visuzalization of our subset via gephi in figure 2:

\begin{figure}[h]
  \centering
  \includegraphics[width=\linewidth]{graph_25.png}
  \caption{Visualization of the chosen subset.}
  \label{fig:minha_figura2}
\end{figure}

\section{Discussion}\label{sec:discussion}
Interpret the structural meaning of your findings:
\begin{itemize}
    \item Existence of hubs and inequality.
    \item Community structure among experts.
    \item Small-world effects and efficiency.
    \item Sociological perspective (motivation, reputation, collaboration).
\end{itemize}

\section{Conclusion}\label{sec:conclusion}
Summarise the findings and highlight what the subset reveals about the interaction patterns of MathOverflow's expert community. Suggest possible future work (temporal analysis, motif analysis, topic-based communities).

%%%%%%%%%%%%%%%%%%%%%%%%%%%%%%%%%%%%%%%%%%%%%%%%%%%%%%%%%%%%%%%%%%%%%%%%%%%%%%%%%%%%%%%%%%%%%%%%%%%%%%%%%%%%%%%%%%%%%%%%

\bibliographystyle{ACM-Reference-Format}
\bibliography{bib}

\end{document}
\endinput
%%
%% End of file `sample-manuscript.tex'.
